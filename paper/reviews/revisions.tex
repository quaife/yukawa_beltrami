\documentclass[11pt]{article}


\usepackage{fullpage}
\usepackage{amsmath,amsfonts,amssymb,stmaryrd}
\usepackage{color}
\newcommand{\comment}[1]{{\color{blue} #1}}
\newcommand{\todo}[1]{{\color{red} #1}}
\newcommand{\note}[1]{{\color{green} #1}}
\newcommand{\grad}{{\triangledown}}
\newcommand{\nn}{{\mathbf{n}}}
\newcommand{\uu}{{\mathbf{u}}}
\newcommand{\xx}{{\mathbf{x}}}
\newcommand{\DD}{{\mathcal{D}}}
\renewcommand{\SS}{{\mathcal{S}}}

\begin{document}

We thank the reviewers for thoroughly reading the manuscript and for
their useful comments.  Below is a list of the revisions that we have
applied to the manuscript.


\section*{Reviewer 1}

\comment{This manuscript presents an integral equation method for
solving the Yukawa-Beltrami equation on the sphere. All the material in
the manuscript are straightforward and fairly standard. That is, the
Green’s formula on a Riemannian manifold is well-known to any
researcher working on PDEs and differential geometry. The jump
relations of layer potentials in Riemannian manifolds are also
well-known (as a local property, it is obvious that the jump relations
are the same for the Yukawa-Beltrami and Laplace-Beltrami operators).
Just to name a reference, Mitrea and Taylor [1] discussed in detail the
boundary layer methods for Lipschitz domains in Riemannian manifolds.
Indeed, the paper actually discussed a much harder problem in greater
generality. The authors should first try to find all relevant work that
has already been done by analysts (for example, do a simple search over
the work by Mitrea and Taylor and check the references therein),
instead of trying to produce unsatisfactory proofs for much simpler and
specific cases. The numerical part of the manuscript uses the Alpert
quadrature for logarithmically singular kernels, which is also quite
standard and well-known.}
\begin{itemize}
  \item We have removed some of the proofs and cited the
  appropriate results from paper [1].
  \item While reference [1] addresses several key results, it is our
  opinion that our manuscript adds a significant contribution to the
  literature.  For instance, the numerical methods that we present, to
  the best of our knowledge, are absent in the literature.  We are also
  unaware of any groups that are using integral equation methods to do
  time stepping for PDEs on surfaces.
  \item Reference [1] does present much more general results then ours.
  However, we are interested in one particular PDE (Yukawa-Beltrami) so
  that we can do time stepping on the surface of the sphere.
  \item While Alpert quadrature is now standard and well-known, it is a
  required tool for us to establish high-order accuracy.  We, in no
  form, are claiming that we are producing new results concerning
  quadrature formula for functions with logarithmic singularities.
 \end{itemize}

\comment{In the introduction part, the authors mentioned very briefly
that the motivation of their work is solving the heat equation on the
sphere.  I would like to see at least three {\bf application} papers
which really need to solve the heat equation on the sphere. Otherwise,
the motivation is not justified and it does not seem to be appropriate
to publish the manuscript since one could then write so many papers
with each particular PDE on a specific manifold without any real
applications.}
\begin{itemize}
  \item In the manuscript, we did motivate the Yukawa-Beltrami equation
  with the isotropic heat equation.  However, the bigger picture is to
  consider PDEs that involve both stiff (diffusion) and non-stiff
  (advection or reaction) terms and then apply an IMEX or
  semi-Lagrangian method.  This has several applications, and 3
  applications are cited in the first sentence of the manuscript.
  \item We added a few sentences to discuss the bigger picture.
\end{itemize}

\comment{All the theoretical results in the manuscript are
straightforward and standard and analysts have proven them in a more
general form. It is strongly recommended that the authors study
previous work done by analysts and cite their work. The authors should
also shorten the manuscript by deleting the “proofs” of Proposition 1,
Lemmas 2 and 3 in the manuscript.}
\begin{itemize}
  \item The proof or Proposition 1 is in Mitrea and Taylor 1999 in
  equation (7.37).  We now reference this equation.
  \item For Lemmas 2 and 3, we now reference Mitrea and Taylor 1999
  Proposition 3.8.
\end{itemize}

\comment{It seems that the kernel of the double layer potential has
high-order logarithmically singular terms even though it is continuous
at the diagonal. Instead of presenting numerical example 1, the authors
should do a little analysis to show that this is indeed true.
Numerical example 2 is also not needed since the operator becomes
increasingly coercive as $k$ increases.}
\begin{itemize}
  \item The reviewer is correct that there is a logarithmic singularity
  in the kernel.  We did do the analysis in the original manuscript in
  Section 2.1.  In particular, equation (11) gives the nature of the
  singularity.
  \item We have left in the first numerical example since it numerically
  verifies our analysis, and also shows that the number of GMRES
  iterations is mesh independent.  It is our opinion that results
  concerning mesh independence should be checked numerically.  It is
  possible that an equation of the form ``Identity + Compact" can be
  solved numerically with mesh independence, but a large number of
  iterations are required before convergence is guaranteed.  This is a
  result that needs to be checked numerically.
  \item We have removed the second numerical example.
\end{itemize}

\comment{The authors need to clarify the real distinction on the
integral equation method for $k > \frac{1}{2}$ and $k \leq
\frac{1}{2}$.  The PDE itself does not seem to have any essential
differences between these two cases. The comment that $k^{2} =
\mathcal{O}(1/\Delta t)$ does not really justify the neglect of the
discussion of the case $k\leq\frac{1}{2}.$}
\begin{itemize}
  \item The reviewer is correct that the fundamental solution defined in
  terms of the Legendre function is valid for all $k > 0$.  However, in
  order to use the Conical function representation (which we do both in
  our analysis and numerical code), we require that $\sqrt{4k^{2}-1}$ is
  real.
  \item We now clarify the fact that the restriction $k > \frac{1}{2}$
  is one of convenience and is not essential.
  \item However, the values of $k$ that we anticipate in our future work
  are much larger than $\frac{1}{2}$.
\end{itemize}

\comment{Lemma 1 is a little unsatisfactory as the authors do not
explain why $\lim_{\theta \rightarrow 0^{+}}F(\theta)$ should be
$\frac{1}{2\pi}$.  Standard approach of finding the constant $C_{k}$
would utilize the fact that the Green’s function satisfies the PDE with
the Dirac delta function as the right hand side and carry out an
integration procedure. The authors could add one sentence or two for
this.}
\begin{itemize}
  \item In equation (7), we define the constant $C_{k}$, but we don't
  justify this value until Section 2.2
  \item Instead of using the standard method of finding the constant
  that guarantees that the PDE applied to the Green's function results
  in the Dirac delta function, we choose the constant so that the
  limiting value of the double-layer potential is $\frac{1}{2\pi}$.
  \item These two methods result in the same constant $C_{k}$ which we
  now state after the proof of Lemma 1.
\end{itemize}

\comment{It seems weird to use $\tilde{V}$ and $\tilde{W}$ instead of
simple $S$ and $D$ to denote the single and double layer potentials,
respectively. Is there any particular reason to do so?  Also, the
justification of single and double layer potentials satisfying the PDE
is totally unnecessary since it uses just simple exchange of order of
differentiation and integration.  The authors should remove it.}
\begin{itemize}
  \item The use of $V$ and $W$ for the single and double layer
  potentials is used in the boundary element method community.  For
  instance, see the works of Hsiao, Wendland, Steinbach, Nigam, etc.
  \item We removed the equations that showed how the layer potentials
  satisfy the PDE.
\end{itemize}

\comment{In summary, the authors should justify the application of the
manuscript and shorten the manuscript (definitely less than ten pages)
in order to get it published. I would recommend the publication of the
manuscript only if the changes are made to accommodate the above
comments.}
\begin{itemize}
  \item The applications are further motivated
  \item We have removed the proofs that were addressed by the reviewer.
  \item While a 10 page paper would be nice, we feel that this is
  unrealistic given the amount of content.  The introduction, numerical
  results, and bibliography take 6 pages alone.  Four pages is not
  sufficient to compute the fundamental solution (in several forms),
  define the layer potentials,  look at their asymptotics, and present
  several results.
\end{itemize}



%%%%%%%%%%%%%%%%%%%%%%%%%%%%%%%%%%%%%%%%%%%%%%%%%%%%%%%%%%%%%%%%%%%%%%%%
\section*{Reviewer 2}
\comment{My only suggestion
has to do with the confusing terminology. A parametrix is usually just
an approximation to a fundamental solution of a partial differential
equation, [1]. In this paper, the terms ``parametrix'' and
``fundamental solution'' are used somewhat interchangeably, e.g.,
Proposition 1 indicates that $G_k$ is a fundamental solution, but it
is referred to as a parametrix in Section 3. The authors might want to
clarify their choice - obviously, strictly at the authors'
discretion. We would also like to point out that similar techniques
have been used to derive fundamental solutions for toroidal
geometries, [2]-[3], and in describing Yukawa scattering, [4].

To summarize, this is an excellent paper, very much acceptable for
publication in ACOM. It requires no further revision.

References:

[1] http://en.wikipedia.org/wiki/Parametrix

[2] Conical functions: http://dlmf.nist.gov/14.19

[3] Toroidal (ring) functions: http://dlmf.nist.gov/14.20

[4] D.I. Fivel. New formulation of dispersion relations for
potential scattering. Phys. Rev., 125(3):1085-1087, 1962.
}

\begin{itemize}
  \item We have clarified that the parametrix that we use for the
  majority of the manuscript is in fact the fundamental solution.

  \item We thank the reviewer for pointing out the three references.
\end{itemize}


\end{document}
